\documentclass[a4paper,12pt,twoside]{article}
\usepackage[ngerman]{babel}
\usepackage[utf8]{inputenc}
\usepackage[T1]{fontenc}
\usepackage[a4paper,margin=2.5cm]{geometry}
\usepackage{fancyhdr}
\pagestyle{fancy}
\fancyhf{} % Kopf- und Fußzeilen leeren
\fancyfoot[C]{\thepage} % Seitenzahl zentriert in der Fußzeile
\usepackage[hidelinks]{hyperref}
\usepackage{listing}

% Titelseite
\title{Ich lerne GO}
\author{Tobi}
\date{\today}

\begin{document}

% Titelseite
\maketitle
\thispagestyle{empty} % Keine Kopf-/Fußzeile auf der Titelseite
\newpage

% Inhaltsverzeichnis
\tableofcontents
\newpage

% Beispielkapitel
\section{Ausführen}
\begin{tabbing}
  \hspace{2mm} \= \hspace{50mm} \= \kill
\>  go run bunga.go \> Lässt das Proglaufen ohne zu compilieren \\
\>  go build bunga.go \> compiliert das Programm \\ 
\>  ./bunga.go \> startet das compilierte Programm \\ 
\>  dlv debug bunga.go \> startet das Programm im Debugger \\
\end{tabbing}

\section{Syntax}
\begin{tabbing}
 \hspace{2mm} \= \hspace{50mm} \= \kill
  \> // \> Kommentare \\
\end{tabbing}
\section{Debugger}
\section{Ein und Ausgabe}
\subsection{Ausgabe}
\begin{tabbing}
 \hspace{2mm} \= \hspace{50mm} \= \kill
 \> \verb|fmt.print("bunga")| \> Ausgabe von Bunga \\
 \> \verb|fmt.print("%#v", per)| \> gibt die Variable per aus \\
 \> \verb|fmt.print("%d", X)| \> gibt den Wert von X aus \\
\end{tabbing}

\section{Variablen}
\begin{tabbing}
 \hspace{2mm} \= \hspace{50mm} \= \kill
 \> var a int \> deklariert die Variabla a als int
\end{tabbing}

\section{Datentypen}
\begin{tabbing}
 \hspace{2mm} \= \hspace{50mm} \= \kill
 \> int \> Ganzzahl, Größe ist plattformabhänging \\
 \> int8, int16, int32, int64 \> Ganzzahl \\
 \> uint8, uint16, uint32, uint64 \> Vorzeichenlose Ganzzahl \\
 \> float32, float64 \> Gleitkommazahl \\ 
 \> bool \> ture oder false \\ 
 \> rune \> int32 \\ 
 \> byte \> uint8 \\
 \> string \> Zeichnkette \\ 
 \> interface{} \> kann jeden Datentypen halten \\
 \> array \> Array feste Länge \\ 
 \> slice \> Array variable Länge \\ 
 \> map \> Hashmap/Dictionary \\ 
 \> struct \> Benutzerdefinierte Struktur \\ 
 \> pointer \> zeigt Adresse einer Variable \\
\end{tabbing}

\section{Schleifen}
\section{Funktionen}
\section{Mutliprozess}
\section{Pakete}
\begin{tabbing}
 \hspace{2mm} \= \hspace{50mm} \= \kill
\> fmt \> Ein und Ausgabe im CLI \\
\end{tabbing}


\section{Installation}
\subsection{Arch Linux}
\begin{itemize}
  \item Compiler \verb| sudo pacman -Sy go |
  \item Debugger \verb| sudo pacman -Sy delve |
  \item LSP \verb| go install golang.org/x/tools/gopls@latest |
  \item Linter \verb| go install github.com/segmentio/golines@latest |
\end{itemize}
\end{document}

