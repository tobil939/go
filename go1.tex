\documentclass[twoside,a4paper,12pt]{article}
\usepackage[ngerman]{babel}
\usepackage[utf8]{inputenc}
\usepackage[T1]{fontenc}
\usepackage{fancyhdr}
\usepackage[hidelinks]{hyperref}
\usepackage{listings}
\usepackage{xcolor}
\usepackage{ulem}
\usepackage{graphicx} % Für Bilder

\usepackage[a4paper, top=2.5cm, bottom=3cm, left=2cm, right=2.5cm]{geometry}

% Catppuccin Macchiato Farbschema für Listings
\definecolor{catppuccinBg}{HTML}{24273A}
\definecolor{catppuccinFg}{HTML}{CAD3F5}
\definecolor{catppuccinRed}{HTML}{ED8796}
\definecolor{catppuccinGreen}{HTML}{A6DA95}
\definecolor{catppuccinYellow}{HTML}{EED49F}
\definecolor{catppuccinBlue}{HTML}{8AADF4}
\definecolor{catppuccinMagenta}{HTML}{F5BDE4}
\definecolor{catppuccinCyan}{HTML}{8BD5CA}

\lstset{
    backgroundcolor=\color{catppuccinBg},
    basicstyle=\ttfamily\color{catppuccinFg},
    keywordstyle=\color{catppuccinBlue},
    commentstyle=\color{catppuccinGreen},
    stringstyle=\color{catppuccinRed},
    numberstyle=\tiny\color{catppuccinYellow},
    breakatwhitespace=false,
    breaklines=true,
    captionpos=b,
    keepspaces=true,
    numbers=left,
    numbersep=5pt,
    showspaces=false,
    showstringspaces=false,
    showtabs=false,
    tabsize=2
}

% Kopf- und Fußzeile
\pagestyle{fancy}
\fancyhf{}
\fancyfoot[C]{\thepage} % Seitenzahl mittig in der Fußzeile
\renewcommand{\headrulewidth}{0.4pt}

\fancyhead[LE,RO]{Ich lerne Go} % Titel links auf linken und rechten Seiten
\fancyhead[LO,RE]{Tobi} % Autor rechts auf linken und rechten Seiten

\title{Ich lerne Go}
\author{Tobi}

\begin{document}

% Titelseite
\maketitle
\thispagestyle{empty} % Keine Kopf-/Fußzeile auf der Titelseite
\newpage

% Inhaltsverzeichnis
\tableofcontents
\newpage

\section{Interaktion}
\subsection{Eingabe}
\begin{tabbing}
  \hspace{2mm} \= \hspace{70mm} \= \kill
  \> \verb| fmt.Scanln | \> Eingabe bis zum Enter \\  
  \> \verb| fmt.Scan | \> Werte getrennt durch Leerzeichen \\ 
  \> \verb| fmt.Scanf | \> Formatierte Eingabe \\ 
  \> \verb| os.Args | \> Ubergabe von Argumenten \\ 
  \> \verb| os.Getenv | \> Einlesen von Umgebungsvariablen \\
  \> \verb| bufio.Reader | \> größere Eingaben oder Zeilenweise \\ 
  \> \verb| bufio.Scanner | \>  Scant Tokens in Text oder Eingabe \\
\end{tabbing}
\subsection{Ausgabe}
\begin{tabbing}
  \hspace{2mm} \= \hspace{70mm} \= \kill
  \> \verb| fmt.Println | \> Ausgabe mit Zeilenumbruch \\ 
  \> \verb| fmt.Print | \> Einfache Ausgabe \\ 
  \> \verb| fmt.Printf | \> Formatierte Ausgabe \\ 
  \> \verb| fmt.Sprintf | \> Ausgabe eines Formatierten Strings \\ 
  \> \verb| fmt.Fprintf | \> schreibt in eine Datei \\ 
  \> \verb| os.Stdout | \> schreibt in eine Datei \\ 
  \> \verb| os.Stderr | \> schreibt eine Fehlermeldung in eine Datei \\ 
\end{tabbing}
\subsection{fmt Steuerzeichen}
\begin{tabbing}
  \hspace{2mm} \= \hspace{70mm} \= \kill
  \> \verb| \n | \> Zeilenumbruch \\ 
  \> \verb| \t | \> Tabular, Einrücken \\ 
  \> \verb| \\ | \> Backslash \\ 
  \> \verb| \" | \> Anführungszeichen \\ 
\end{tabbing}
\subsection{Plotten}

\section{Syntax}
\begin{tabbing}
  \hspace{2mm} \= \hspace{70mm} \= \kill
  \> \verb| // | \> Kommentar in einer Zeile \\
  \> \verb| /*   */ | \> Kommmentar über mehrere Zeilen \\ 
  \> \verb| () | \> Bedingungen, Definitionen, Funktionsaufrufe \\ 
  \> \verb| {} | \> Code-Blöcke \\ 
  \> \verb| : | \> Kurzdeklarationen \\ 
  \> \verb| ; | \> Trennung von z.B. Argumenten \\ 
  \> \verb| " " | \> Strings \\ 
  \> \verb| ' ' | \> Runes \\ 
\end{tabbing}

\section{Abfragen}
\subsection{if}
\begin{center}
  \begin{minipage}{1.0\textwidth}
    \begin{lstlisting}[language=Go]
    if Bedingung1 {
      // Wenn Bedingung1 erfuellt
    } else if Bedingung2 {
      // Wenn Bedingung2 erfuellt
    } else {
      // Wenn keine Bedingung erfuellt
    }
    \end{lstlisting}
  \end{minipage}
\end{center}

\subsection{switch}
\begin{center}
  \begin{minipage}{1.0\textwidth}
    \begin{lstlisting}[language=Go]
    switch tag {
    case "Mittwoche":
      // Was wenn es Mittwoch ist? 
    case "Donnerstag":
      // Was wenn es Donnerstag ist?
    case "Freitag", "Samstag", "Sonntag":
      // Verlaengertes Wochenende 
    default:
      // Wenn Kein Tag ist
    }
    \end{lstlisting}
  \end{minipage}
\end{center}

\subsection{typ-switch}
\begin{center}
  \begin{minipage}{1.0\textwidth}
    \begin{lstlisting}[language=Go]
    switch t := t.(type) {
    case bool:
     // Wenn t bool
    case int:
      // Wenn t int 
    case *bool:
      // Wenn t der Pointer einer bool 
    default: 
      // Datentyp von t nicht dabei
    } 
    \end{lstlisting}
  \end{minipage}
\end{center}

\subsection{select}
\begin{center}
  \begin{minipage}{1.0\textwidth}
    \begin{lstlisting}[language=Go]
    select {
    case msg1 := <-chl1:
      // Nachricht von Prozess1 oder Kanal1 
    case msg2 := <-chl2:
      // Nachricht von Prozess2 oder Kanal2
    default:
      // Opriton wenn keine Nachricht, default, 
      // sollte verwendet werden, muss aber nicht
    }
    \end{lstlisting}
  \end{minipage}
\end{center}

\section{Schleifen}
\subsection{for}
\begin{center}
  \begin{minipage}{1.0\textwidth}
    \begin{lstlisting}[language=Go]
    for Initialisierung; Bedingung; Nachbearbeitung {
      // Schleifeninhalt
    }
    \end{lstlisting}
  \end{minipage}
\end{center}
\textbf{Initialisierung:} Initialisert eine oder mehrere Variablen, \\ 
Die in der Schleife verwendet werden, wird nur am Anfang der Schleife gemacht \\ 
\textbf{Bedingung:} wird vor jeder Iteration ausgewertet, bei true wird \\ 
der Inhalt der Schleife ausgeführt \\ 
\textbf{Nachbearbeitung:} wird am Ende jeder Iteration gemacht, \\ 
z.B. hochzählen des Zählers \\ 

\subsection{for mit break}
\begin{center}
  \begin{minipage}{1.0\textwidth}
    \begin{lstlisting}[language=Go]
    for i := 0; i < 10; i++ {
      if i == 5 {
        break // Schleife wird beendet, wenn 5 erreicht ist
      }
      // Iterationen 0 bis 4
    }
    \end{lstlisting}
  \end{minipage}
\end{center}

\subsection{for mit continue}
\begin{center}
  \begin{minipage}{1.0\textwidth}
    \begin{lstlisting}[language=Go]
    for i := 0; i < 10; i++ {
      if i == 5 {
        continue // Aktion der Schleife wird ueberprungen wenn i = 5 
      }
      // Iterationen 0 bis 4; 6 bis 10
    }
    \end{lstlisting}
  \end{minipage}
\end{center}

\subsection{for mit range}
\begin{center}
  \begin{minipage}{1.0\textwidth}
    \begin{lstlisting}[language=Go]
    for index, value := range collection {
      // Schleifeninhalt
    }

    // wenn nur Value benoetigt
    for _, value := range collection { 
      // Schleifeninhalt
    }

    // wenn nur der Index benoetigt 
    for index := range collection {
      // Schleifeninhalt 
    }
    \end{lstlisting}
  \end{minipage}
\end{center}
\textbf{index:} Dies ist der Index, oder die Position des aktuellen Elements \\ 
\textbf{value:} Dies ist das aktuelle Element \\ 
\textbf{collection:} Dies ist die Sammlung in der das Element ist \\

\subsection{while pseudo}
\begin{center}
  \begin{minipage}{1.0\textwidth}
    \begin{lstlisting}[language=Go]
    sum := 1 
    for sum < 1000 { 
      sum += sum
    }
    \end{lstlisting}
  \end{minipage}
\end{center}

\section{Operatoren}
\subsection{Standard}
\subsection{Mathematisch}
\subsection{Logisch}

\section{Variablen}
\subsection{Variablen}
\subsection{Umwandeln}
\subsection{Global}
\subsection{Lokal}

\section{Mathematik}
\subsection{Standard}
\subsection{Exponential}
\subsection{Vektor}
\subsection{Trigonometrie}

\section{Datenmanagement}
\subsection{Speichern}
\subsection{Laden}
\subsection{Einlesen}
\subsection{Parsen}

\section{Speichermanagement}

\section{Fehlermanagement}
\subsection{Exception}
\subsection{Error}

\section{Regeln}

\section{Multiprozess}

\section{Bibliotheken}
\subsection{fmt}

\subsection{os}

\subsection{bufio}

\section{Ausführen}
\subsection{Kompilieren/Ausführen}
\begin{tabbing}
  \hspace{2mm} \= \hspace{50mm} \= \kill
  \> go run bunga.go \> führt das Programm aus, ohne kompilieren \\ 
  \> go build bunga.go \> kompiliert das Programm \\ 
  \> ./bunga.go \> führt das kompilierte Programm aus \\ 
  \> \verb|.\bunga.exe| \> führt das kompilierte Programm aus (Windows) \\
  \> Cross-Kompilierung, erstellt unter Linux ein Programm für Windows \\
  \> GOOS=windows GOARCH=amd64 go build bunga.go \\
  \> GOOS=linux GOARCH=amd64 go build bunga.go \\ 
  \> GOOS=darwin GOARCH=amd64 go build bunga. go (macOS)\\
\end{tabbing}
\subsection{Debugging}
\begin{tabbing}
  \hspace{2mm} \= \hspace{50mm} \= \kill
  \> dlv debug bunga.go \> starten das Programm im Debugger \\
\end{tabbing}

\newpage
\section{Hello World!}
\begin{center}
  \begin{minipage}{1.0\textwidth}
    \begin{lstlisting}[language=Go]
      pacman main

      import "fmt"

      func main(){
        fmt.Println("Hello World!")
      }
    \end{lstlisting}
  \end{minipage}
\end{center}

\section{Installation}
\subsection{Kompiler/Interpreter}
\subsubsection{Arch Linux}
\begin{tabbing}
  \hspace{2mm} \= \hspace{30mm} \= \kill
  \> Kompiler \> \verb|sudo pacman -Sy go| \\ 
  \> überprüfen \> \verb|go version| \\
\end{tabbing}
\subsubsection{Windows}
\begin{tabbing}
  \hspace{2mm} \= \hspace{30mm} \= \kill
  \> Kompiler \> \href{https://go.dev/doc/install}{Link zu golang.org} \\
  \> überprüfen \> \verb|go version| \\
\end{tabbing}
\subsection{Debugger}
\subsubsection{Arch Linux}
\begin{tabbing}
  \hspace{2mm} \= \hspace{30mm} \= \kill
  \> Debugger \> \verb|sudo pacman -Sy delve| \\
\end{tabbing}
\subsubsection{Windows}
\begin{tabbing}
  \hspace{2mm} \= \hspace{30mm} \= \kill
  \> Debugger \> \verb|go install github.com/go-delve/delve/cmd/dlv@latest| \\
\end{tabbing}
\subsection{LSP}
\begin{tabbing}
  \hspace{2mm} \= \hspace{30mm} \= \kill
  \> LSP \>  \verb|go install golang.org/x/tools/gopls@latest| \\
  \> Linter \> \verb|go install github.com/segmentio/golines@latest| \\
\end{tabbing}
\subsection{Bibliothekeninstaller}
\begin{tabbing}
  \hspace{2mm} \= \hspace{30mm} \= \kill
  \> Go-Pakete \> \verb|go get -u github.com/username/repository|
 \end{tabbing}
\subsection{Sonstiges}
\subsubsection{Umgebungsvariablen Windows}
\begin{itemize}
  \item GOPATH: Standardmäßig \verb|C:\Users\<Benutzer>\go|
  \item GOROOT: Standardmäßig der Installationspfad von Go
  \item PATH: Enthält \verb|%GOPATH%\bin| und \verb|%GOROOT%\bin|
\end{itemize}

\end{document}

%# ==========  
%# ==========  
%# ==========  
%# ==========  
%# ==========  
\subsection{tabbing}
\begin{tabbing}
  \hspace{2mm} \= \hspace{50mm} \= \kill
  
\end{tabbing}

\subsection{Durchgestrichene Zeile}
Dies ist eine \sout{durchgestrichene} Zeile.

\subsection{Links}
Hier ist ein \href{https://github.com}{Link zu GitHub}.

\begin{center}
  \begin{minipage}{0.9\textwidth}
    \begin{lstlisting}[language=Go]
    package main

    import "fmt"

    func main() {
      fmt.Println("Hallo, Welt!")
    }
    \end{lstlisting}
  \end{minipage}
\end{center}

\subsection{Bilder}
%\begin{figure}[h]
%    \centering
%    \includegraphics[width=0.5\textwidth]{deinbild} % Ersetze "deinbild" durch den Namen deiner Bilddatei
%    \caption{Ein Beispielbild}
%    \label{fig:beispielbild}
%\end{figure}

\subsection{Tabellen}
\begin{table}[h]
    \centering
    \begin{tabular}{|c|c|c|}
        \hline
        A & B & C \\
        \hline
        1 & 2 & 3 \\
        \hline
        4 & 5 & 6 \\
        \hline
    \end{tabular}
    \caption{Eine Beispiel-Tabelle}
    \label{tab:beispieltabelle}
\end{table}


