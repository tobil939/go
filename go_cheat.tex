\documentclass[a4paper,10pt]{article}
\usepackage[a4paper,margin=1cm]{geometry}
\usepackage{multicol}
\usepackage{fancyhdr}
\usepackage{xcolor}
\usepackage{listings}
\usepackage{titlesec}

% Farben für Code
\definecolor{codebg}{RGB}{245,245,245}
\definecolor{keyword}{RGB}{0,0,180}
\definecolor{comment}{RGB}{0,128,0}
\definecolor{string}{RGB}{163,21,21}

\lstset{
  backgroundcolor=\color{codebg},
  basicstyle=\ttfamily\footnotesize,
  keywordstyle=\color{keyword}\bfseries,
  commentstyle=\color{comment},
  stringstyle=\color{string},
  numbers=left,
  numberstyle=\tiny,
  breaklines=true,
  frame=single,
  showstringspaces=false,
  tabsize=4
}

% Layout für die Überschriften
\titleformat{\section}{\large\bfseries}{}{0em}{}
\titleformat{\subsection}{\normalsize\bfseries}{}{0em}{}

% Kopf- und Fußzeile
\pagestyle{fancy}
\fancyhf{}
\lhead{Go Cheat Sheet}
\rhead{\thepage}
\cfoot{\small Generated by ChatGPT}

\begin{document}

\begin{center}
  {\LARGE \textbf{Go Cheat Sheet}} \\[1em]
  {\large A quick guide to the Go programming language (Golang)}
\end{center}

\vspace{0.5cm}

\begin{multicols*}{2}

% Section 1: Basics
\section{Basics}
\subsection{Hello, World}
\begin{lstlisting}[language=Go]
package main

import "fmt"

func main() {
    fmt.Println("Hello, World!")
}
\end{lstlisting}

\subsection{Variables}
\begin{lstlisting}[language=Go]
var x int = 10
var y = "Hello"
z := 3.14 // Short declaration
\end{lstlisting}

\subsection{Data Types}
\begin{itemize}
  \item Basic: \texttt{int}, \texttt{float64}, \texttt{string}, \texttt{bool}
  \item Complex: \texttt{array}, \texttt{slice}, \texttt{map}, \texttt{struct}
  \item Special: \texttt{interface\{\}}, \texttt{chan}, \texttt{pointer}
\end{itemize}

\subsection{Constants}
\begin{lstlisting}[language=Go]
const Pi = 3.14
const (
    A = 1
    B = "Hello"
)
\end{lstlisting}

\subsection{Control Structures}
\begin{lstlisting}[language=Go]
if x > 10 {
    fmt.Println("x is large")
} else {
    fmt.Println("x is small")
}

for i := 0; i < 5; i++ {
    fmt.Println(i)
}

switch day := 3; day {
case 1:
    fmt.Println("Monday")
case 2:
    fmt.Println("Tuesday")
default:
    fmt.Println("Another day")
}
\end{lstlisting}

% Section 2: Functions
\section{Functions}
\subsection{Basic Function}
\begin{lstlisting}[language=Go]
func add(x int, y int) int {
    return x + y
}
\end{lstlisting}

\subsection{Multiple Return Values}
\begin{lstlisting}[language=Go]
func swap(a, b string) (string, string) {
    return b, a
}
\end{lstlisting}

\subsection{Anonymous Functions}
\begin{lstlisting}[language=Go]
sum := func(a, b int) int {
    return a + b
}
fmt.Println(sum(3, 4))
\end{lstlisting}

% Section 3: Data Structures
\section{Data Structures}
\subsection{Arrays and Slices}
\begin{lstlisting}[language=Go]
var arr [3]int = [3]int{1, 2, 3}
slice := []int{4, 5, 6}
\end{lstlisting}

\subsection{Maps}
\begin{lstlisting}[language=Go]
m := map[string]int{"a": 1, "b": 2}
fmt.Println(m["a"])
\end{lstlisting}

\subsection{Structs}
\begin{lstlisting}[language=Go]
type Person struct {
    Name string
    Age  int
}
p := Person{Name: "Alice", Age: 25}
fmt.Println(p.Name)
\end{lstlisting}

% Section 4: Concurrency
\section{Concurrency}
\subsection{Goroutines}
\begin{lstlisting}[language=Go]
go func() {
    fmt.Println("Hello from a goroutine")
}()
\end{lstlisting}

\subsection{Channels}
\begin{lstlisting}[language=Go]
ch := make(chan int)
go func() {
    ch <- 42
}()
fmt.Println(<-ch)
\end{lstlisting}

\subsection{WaitGroup}
\begin{lstlisting}[language=Go]
import "sync"
var wg sync.WaitGroup
wg.Add(1)
go func() {
    fmt.Println("Done")
    wg.Done()
}()
wg.Wait()
\end{lstlisting}

% Section 5: Error Handling
\section{Error Handling}
\begin{lstlisting}[language=Go]
func divide(a, b int) (int, error) {
    if b == 0 {
        return 0, fmt.Errorf("division by zero")
    }
    return a / b, nil
}

result, err := divide(4, 2)
if err != nil {
    fmt.Println("Error:", err)
} else {
    fmt.Println("Result:", result)
}
\end{lstlisting}

\end{multicols*}

\end{document}


